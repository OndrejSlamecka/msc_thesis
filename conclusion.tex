\chapter{Conclusion}

We introduced Markov decision processes,
the problem of their verification and known approaches to its solution.
Monte Carlo tree search was described in its most common variant UCT
(Upper Confidence bound applied to Trees),
together with its applications to maximizing rewards in MDPs and solving
games.
We suggested various new algorithms by combining the known
approaches to MDP verification with the techniques of MCTS. These
algorithms were implemented and evaluated on standard and new models.

We have observed the MAX-DIFF variant of BRTDP is a powerful
heuristic which itself often balances well between exploration and
exploitation in common MDP models.
%Such behaviour was our goal when designing the MCTS based algorithm.
Our MCTS-BRTDP algorithm performs only slightly worse on the PRISM
benchmark suite while it performs significantly better on many models
which are hard for BRTDP.  Even though value iteration does well on
those hard-for-BRTDP models, it loses on the PRISM suite, making
MCTS-BRTDP a universal choice for any MDP.

There is still a lot of work to be done in order to properly understand
how MCTS based methods may be applied in MDP verification. A
quantitative study of the algorithms' executions would help understand
which parts
of an MDP are explored even though they are not important and which
important parts are explored too late. Such observations could be used
to suggest new formulas for tree node selection or other variations,
however, due to the complexity of the models, this might be a demanding task.

Another interesting area of research might be into
new algorithms where the MCTS approach has better chances to
improve the search, for example, one might try running MCTS and BRTDP in
stages, each time for a limited number of iterations until the bounds
are sufficiently close. Interleaving of MCTS-BRTDP-VCB (to get a precise
lower bound quickly) with BRTDP (to eliminate overly optimistic upper
bounds) might also work for some MDPs.

There are also rather easy practical tasks like adding support for
time-bounded properties or extracting the strategy from the solution.
%\footnote{Once
%the algorithm converges choose the action with the best upper bound at
%each step.}.
