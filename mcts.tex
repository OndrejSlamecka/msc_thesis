\chapter{Monte Carlo Tree Search}
\label{ch_mcts}

Monte Carlo methods\footnote{Not to be confused with
Monte Carlo algorithms which are precisely defined as the algorithms
solving the decision problems in classes BPP and RP.} are using random sampling to estimate the correct
solution to a problem. The first serious use of Monte Carlo methods was
by Stanislaw Ulam and John von Neumann during their work on the
Manhattan project, but the technique has since spread into many areas of
science due to its general applicability.

One of the celebrated Monte Carlo methods
is the simulated annealing algorithm (so called due to its
origin in statistical physics) which is an improved version of
Metropolis algorithm (invented by a Manhattan project scientist
Nicholas C. Metropolis and others).

This method found its way into game
theory in 1993 when it was applied to the board game Go
\parencite{MonteCarloGo}. The approach was later further improved
\parencite{MonteCarloGoDevel} but the real breakthrough came in 2006
when Coulom \parencite{Coulom} and Kocsis, Szepesvári \parencite{Kocsis}
independetly explored the idea of maintaing a tree which would guide the
search for strategies -- thus discovering Monte Carlo Tree Search.
This was eventually used in the AlphaGo program
\parencite{AlphaGo}, the first
computer program to beat professional human players.

Monte Carlo Tree Search (MCTS) is, in short, a
heuristic search algorithm for finding good strategies in complex
decision processes by combining standard approaches of artificial
intelligence and computational statistics: tree search and sampling.

In this chapter the general MCTS scheme is defined and a concrete instance
called UCT is shown together with its application to maximizing rewards
in MDPs and games. The chapter is based mostly on a thorough MCTS survey
paper \parencite{mcts_survey}.

Since the research into MCTS focuses mainly on using MCTS for reward maximization
we also focus on that in this chapter unlike the rest of the thesis.

TODO: define MDP with rewards here (but mention it in the mdp chapter
too, just briefly, informaly)

\section{General MCTS Scheme}

MCTS iteratively builds a tree which approximates possible resulting
rewards of strategies in the decision process. In each iteration the
tree guides the search to balance between exploitation of known good
strategies and exploration of new strategies. When the search leaves the
tree it proceeds at random and upon terminating it adds a new leaf to
the tree and updates its ancestors with the result. This is summarized
in \autoref{mcts}.


\begin{algorithm}
\caption{General Monte Carlo Tree Search method}
\label{mcts}
\begin{algorithmic}
\Function{MCTS}{$s_0$}
    \State Let $v_0$ be the root of the MCTS tree, with $v_0.state = s_0$.
    \While{within computational budget}
        \State $v_l \gets \Call{TreePolicy}{v_0}$
        \State $\Delta \gets \Call{DefaultPolicy}{v_l}$
        \State $\Call{Backup}{v_l, \Delta}$
    \EndWhile
    \State \Return Action from $v_0$ to the best node (by some
    given metric).
\EndFunction
\end{algorithmic}
\end{algorithm}

\section{Upper Confidence Bound for Trees}

The most common implementation of the general scheme is {\em Upper
Confidence Bound for Trees} (UCT) which utilizes formula \ref{UCB}.
In this formula $\overline{X}_i$ represents the expected outcome
from node $i$, $n$ is the number of visits to all nodes, $n_i$ is the
number of visits to node $i$. $C$ is an arbitrary constant.

\begin{equation}
\label{UCB}
UCT_i = \overline{X}_i + C \sqrt{ \frac{2 \ln n}{n_i} }
\end{equation}

Selecting a tree node which maximizes this value makes the algorithm balance
between exploitation of known good strategies and exploration of new.
Constant $C$ instructs the algorithm how much weight to give to
exploration. TODO: Mention the result of Kocsis and Sepe... which proves
$1/\sqrt(2)$ to be optimal under some circumstances.

\autoref{uct} is implementation of the general \autoref{mcts}. As UCT is
the most common type of MCTS algorithm, the terms are often used
interchangeably in literature.

TODO: Define precisely the problem uct is solving (it should be
maximization of rewards in MDP).

\begin{algorithm}
    \caption{Upper Confidence Bound for Trees}
\label{uct}
\begin{algorithmic}
\Function{UCT}{$s_0$}
    \State TODO
    \State Let $v_0$ be the root of the MCTS tree, with $v_0.state = s_0$.
    \While{within computational budget}
        \State $v_l \gets \Call{TreePolicy}{v_0}$
        \State $\Delta \gets \Call{DefaultPolicy}{v_l}$
        \State $\Call{Backup}{v_l, \Delta}$
    \EndWhile
    \State \Return Action from $v_0$ to the best node (by some
    given metric).
\EndFunction
\end{algorithmic}
\end{algorithm}

\section{Solving Games}

In this section a brief introduction to game theory is given, starting
with a definition of games, strategies and solutions to games. We
proceed by presenting the standard {\em minimax} algorithm, then showing
how MCTS can be used to solve games and how it compares with minimax.
Lastly applications to concrete games are shown, most notably the game
of Go which has been a fruitful subject of research on MCTS.

Our definition of a game is that of a {\em perfect-information
extensive-form game}. The perfect-information in games corresponds to
full observability in MDP.  The reader can notice other similarities
with MDPs as well, for example games have states, final states, actions
and enabled actions, as well as rewards. On the other hand the
transition function is deterministic.

\begin{definition}
    A {\em game} is a tuple $G = (N, S, F, A, E, \Delta, \rho, u)$,
    where $N \subseteq \mathbb{N}$ is a finite set of players,
    $S$ is a set of states,
    $F \subseteq S$ is a set of final (terminal) states,
    $A$ is the set of actions,
    $E : S \to \mathcal{P}^+(A)$ is a function which tells the player
    which non-empty set of actions can be played in a given state,
    $\Delta : S \times A \to A$ is the transition function,
    $\rho : S \to N$ is a function which determines who plays in each
    state, and finally $u = (u_1,\ldots,u_n)$ is the tuple of
    utility (reward) functions where each $u_i, i \in N$ is a function
    $u_i : F \to \mathbb{R}$.
\end{definition}

TODO: there may be some more conditions I didn't realize in the definitions, check
ia168

Game starts in a distinguished state $s_0$ and players take turns until
a terminal state (in $F$) reached. In a turn a player in state $s$ chooses
action $a$ from $E(s)$ and the action leads to state $\Delta(s,a)$.
If the state $\Delta(s,a)$ is terminal, then every player $i$ receives
the reward $u_i(\Delta(s,a))$.

A common choice of the utility function is $+1, 0, -1$ for victory, draw
and loss, respectively.

\begin{definition}
    A {\em strategy} of player $i$ in a game $G$ is a function
    $\pi_i : S \to \distribution{A}$, such that for all $s \in S$ and $a
    \in A$ it holds that if $\pi_i(s)(a) > 0$ then $a \in E(s)$.

    A {\em strategy profile} is an $N$-tuple with a strategy for each player.
\end{definition}

We assume the players are rational and thus pick strategies which
optimize for their goals. A goal may be to maximize the player's reward,
another goal may be to get a higher score than the opponents.

TODO: example to show a difference between the goals (ideally a game
where both goals can be achieved but if Alice beats Bob she loses 5 gold
if she optimized for reward but if she optimizes for reward she has
lower profit than Bob).

A famous class of strategic profiles are the so called Nash equilbria.
A strategic profile is a Nash equilibrium, if no player can get a higher
reward by switching a strategy. (TODO: some history)

\subsection{Zero-sum games and minimax}

In zero-sum games one player wins at the cost of the other losing.
There are plenty of examples of such games, most notable are Chess and Go.

\begin{definition}
    A game is {\em zero-sum} if the rewards always sum to zero.
    That is $\sum_{i \in N} u_i(s) = 0$ for all $s \in F$.
\end{definition}

TODO: Explain minimax alg. (mention it was John von Neumann, again)

TODO: Explain minimax solutions are precisely Nash equilibria on zero sum
games.

\subsection{Solving with MCTS}

TODO: How to modify UCT to solve games (negamax backup).

TODO: When to choose minimax and when MCTS.
Mention (or prove?) that mcts converges to minimax.

TODO: Show an example.

\subsection{Computer Go}

TODO: How is MCTS used in some simple Go program.

TODO: The gist of AlphaGo.

TODO: other games?
