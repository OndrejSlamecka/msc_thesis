\documentclass[
  digital, %% This option enables the default options for the
           %% digital version of a document. Replace with `printed`
           %% to enable the default options for the printed version
           %% of a document.
  table,   %% Causes the coloring of tables. Replace with `notable`
           %% to restore plain tables.
  lof,     %% Prints the List of Figures. Replace with `nolof` to
           %% hide the List of Figures.
  lot,     %% Prints the List of Tables. Replace with `nolot` to
           %% hide the List of Tables.
  %% More options are listed in the user guide at
  %% <http://mirrors.ctan.org/macros/latex/contrib/fithesis/guide/mu/fi.pdf>.
]{fithesis3}
%% The following section sets up the locales used in the thesis.
%\usepackage[resetfonts]{cmap} %% We need to load the T2A font encoding
\usepackage[T1]{fontenc}  %% to use the Cyrillic fonts with Russian texts.
\usepackage[
  main=english,  %% By using `czech` or `slovak` as the main locale
                 %% instead of `english`, you can typeset the thesis
                 %% in either Czech or Slovak, respectively.
  english, czech %% The additional keys allow
]{babel}         %% foreign texts to be typeset as follows:
%%
%%   \begin{otherlanguage}{german}  ... \end{otherlanguage}
%%   \begin{otherlanguage}{russian} ... \end{otherlanguage}
%%   \begin{otherlanguage}{czech}   ... \end{otherlanguage}
%%   \begin{otherlanguage}{slovak}  ... \end{otherlanguage}
%%
%% For non-Latin scripts, it may be necessary to load additional
%% fonts:
% \usepackage{paratype}
% \def\textrussian#1{{\usefont{T2A}{PTSerif-TLF}{m}{rm}#1}}
%%
%% The following section sets up the metadata of the thesis.
\thesissetup{
    date          = \the\year/\the\month/\the\day,
    university    = mu,
    faculty       = fi,
    type          = mgr,
    author        = Ondřej Slámečka,
    gender        = m,
    advisor       = {doc. RNDr. Tomáš Brázdil, Ph.D.},
    title         = {Monte Carlo Tree Search in Verification
of Markov Decision Processes},
    TeXtitle      = {Monte Carlo Tree Search \\ in Verification
of Markov Decision Processes},
    keywords      = {Monte Carlo Tree Search, Markov Decision Process,
    Verification, Reachability, Learning Algorithm, Heuristic},
    TeXkeywords   = {Monte Carlo Tree Search, Markov Decision Process,
    Verification, Reachability, Learning Algorithm, Heuristic},
    abstract      = {We explore how Monte Carlo tree search type algorithms can be
        used for verifictation of Markov decision processes (with
        complete information) by balancing between exhaustive search
        and heuristic search (which might spend a long time in local
        minima).
        Several new algorithms either based on UCT (most successful MCTS
        algorithm) or its variations are proposed and
        experimentally evaluated on standard models. Our results show
        MCTS type algorithms perform as well as BRTDP on most standard
        models but are faster on models where BRTDP underestimates
        certain paths to goals.
        },
    thanks        = {These are the acknowledgements for my thesis, which can

                     span multiple paragraphs.},
    bib           = {thesis.bib}
}
\usepackage{makeidx}      %% The `makeidx` package contains
\makeindex                %% helper commands for index typesetting.
%% These additional packages are used within the document:
\usepackage{paralist} %% Compact list environments
\usepackage{amsmath}  %% Mathematics
\usepackage{amsthm}
\usepackage{amsfonts}
\usepackage{mathtools}
\usepackage{pdflscape}
\usepackage{tikz}
\usepackage{url}      %% Hyperlinks
\usepackage{listings} %% Source code highlighting
\lstset{
  basicstyle      = \ttfamily,%
  identifierstyle = \color{black},%
  keywordstyle    = \color{blue},%
  keywordstyle    = {[2]\color{cyan}},%
  keywordstyle    = {[3]\color{olive}},%
  stringstyle     = \color{teal},%
  commentstyle    = \itshape\color{magenta}}
\usepackage{floatrow} %% Putting captions above tables
\floatsetup[table]{capposition=top}

%% Algortihms
\usepackage{algorithm} % this does the figure environment
\usepackage{algpseudocode} % algorithmicx, this does the actual algs

%% Theorems
\theoremstyle{definition}
\newtheorem{definition}{Definition}
\newtheorem{example}{Example}
\providecommand*{\exampleautorefname}{Example}

%% Commands
\newcommand{\powerset}[1]{\mathcal{P}(#1)}
\newcommand{\distribution}[1]{\mathcal{D}(#1)}
\DeclareMathOperator{\last}{last}

%% Some useful commands for pseudocode
\algdef{SE}[DOWHILE]{Do}{doWhile}{\algorithmicdo}[1]{\algorithmicwhile\ #1}%
\algnewcommand\algorithmicforeach{\textbf{for each}}
\algdef{S}[FOR]{ForEach}[1]{\algorithmicforeach\ #1\ \algorithmicdo}

%% TikZ, initial states, clouds and such
\usetikzlibrary{arrows,automata,shapes}


\begin{document}


\chapter{Introduction}

%A rover called Curiosity landed on Mars in August 2012 and is performing
%research in the Gale crater since then. It has to make decisions with
%uncertain consequences to maximize its scientific output but also to maintain
%operational state.

Zero-configuration networking protocol is used to automatically set up
computer networks without the need for external help. The computers
select their IP addresses in the created network randomly with the goal
to achieve a working configuration.

Another protocol is IEEE 802.11 Wireless LAN for communication
of wireless network devices. Part of the protocol describes what to do
when two signals are sent simultaneously and collide. The senders then
use random delays to avoid subsequent collisions.

%Scientists models these and many similar real-world problems with {\em
%Markov Decision Processes} (MDPs). The participants in these processes take
%actions of their choice but the results of these actions are not
%certain -- the rover may finish an experiment with high probability but
%also break with small, the network may be configured with high
%probability or not with small.

Scientists model these and many similar real-world problems with {\em
Markov Decision Processes} (MDPs). Controllers in these processes take
actions of their choice but the results of these actions are not
certain -- in a zero-conf network a chosen IP address might be available
with high probability or taken with small, a signal in a wireless LAN
might get delivered or it might collide with another signal.

Various properties of Markov Decision Processes have been well studied
in the last seventy years. The usual target is to attain the highest
reward from the process, e.g., research as much as possible on Mars
\parencite{MarsRover}.
However, in many situations, it is desirable to learn what is the
probability the process succeeds (network
configures) or fails (signal is not delivered) --
to encourage actions increasing the likelihood of success,
or avoid them for the negative case.
%to avoid taking such
%actions, prevent adversaries from forcing them or, for the positive
%case, to encourage them.

Classical algorithms for verification (proving properties) of MDPs
based on dynamic programming can be used for finding the
maximum probability of success or failure (maximum over all possible decisions in
each step -- strategies). Under some circumstances, these algorithms are
very good but recently it has been shown that learning based methods
like BRTDP can outperform them on many MDPs \parencite{atva14}.

Monte Carlo Tree Search is a heuristic search algorithm which has been
successfully used to find strategies with high rewards in Markov decision
processes. Recently the algorithm had a big success in the field of
computer Go.

\pagebreak

In this thesis, we explore how Monte Carlo Tree Search can be used to
find strategies maximising given properties. Three algorithms are
suggested: one a variation of Upper Confidence Tree (UCT) algorithm (a variant
of MCTS), one a fusion of UCT and BRTDP and the last one a BRTDP variant
using a part of the UCT algorithm idea.  Measurements show that these
algorithms have an advantage on several models.

The thesis is structured as follows. After this introduction, the second
chapter is devoted to Markov Decision Processes and prior work on their
verification.
The third chapter describes Monte Carlo Tree Search and its application
to maximising rewards in MDPs and games. In the fourth chapter our new
algorithms are described. In the fifth chapter,
the algorithms are evaluated on models available with the PRISM project
as well as models newly created specifically for better comparison.
The sixth chapter concludes the
results with suggestions for future work.

The results in the fourth and fifth chapters are original results of
a collaboration of Pranav Ashok, Tomáš Brázdil, Jan Křetínský and the
author of this thesis. The MCTS-BRTDP algorithm was suggested by the
author of this thesis and the BMCTS and BRTDP-UCB algorithms by Pranav
Ashok. The algorithms were implemented as part of the PRISM model
checker mostly when pair programming with Pranav Ashok.

\chapter{Markov Decision Processes}
\label{ch_mdp}

At first, we look at Markov chains, a simpler type of probabilistic processes
which do not offer choices to be made.  Then we generalize to Markov Decision
Processes.  After that, we introduce the problem of {\em verification} (checking
if a given MDP satisfies a given property), overview known solutions with a
focus on value iteration and Bounded Real-Time Dynamic Programming (which will be
used as a part of new algorithms in later chapters).

It is assumed the reader is familiar with basic notions of probability
theory, namely those of {\em probability space} and {\em probability measure}.
Function $f : X \to [0,1]$ is a {\em probability distribution} over
a countable set $X$ if $\sum_{x \in X} f(x) = 1$.
$\distribution{S}$ denotes the
set of probability distributions on a set $S$.
Usually the probability space $(\Omega, \mathcal{F}, P)$
is implicitly known and we use $P(E)$ to
denote the probability of an event $E$.
We refer the reader to
\parencite{probability} for a proper treatment of probability theory.

\section{Markov Chains}

Markov chains are a simpler formalism than
Markov decision process and provide a good first intuition about
probabilistic models. In a Markov chain, there are no decisions, only
probabilities of transition.

\begin{definition}
    Let $S$ be a finite set of states.
    A sequence of random variables $(X_i : \Omega \to S)^{\infty}_{i=0}$
    is called a {\em finite discrete-time time-homogeneous Markov chain}
    if the probability of moving to a state is given only by the
    current state, that is if
    there exists a matrix $p$ such that for any $k$
    \begin{align*}
    P(X_{k+1} = s_{k+1} \mid X_k = s_k, X_{k-1} = s_{k-1}, \ldots, X_{1} = s_{1}) = \\
    = P(X_{k+1} = s_{k+1} \mid X_k = s_k) = p_{s_{k},s_{k+1}}
    \end{align*}
\end{definition}

In many applications it is natural to have an initial state
$s$ ($X_0 = s$) and a chain can
then be drawn as a graph with probabilistic transitions. Note
that the weights of the outgoing edges from a node sum to one, as the
transition probabilities form a distribution on the set of states (for
simplicity we do not draw the transitions with zero probability).

\begin{example}
    \label{ex_mc}
    In this example $s_0$ is the initial state,
    $P(X_2 = s_1 \mid X_1 = s_0) = 0.5$,
    $P(X_2 = s_2 \mid X_1 = s_0) = 0.25$,
    $P(X_2 = s_0 \mid X_1 = s_0) = 0.25$,
    $P(X_{i+1} = s_1 \mid X_{i} = s_1) = 1$,
    $P(X_{i+1} = s_2 \mid X_{i} = s_2) = 1$,
    for all $i \in \mathbb{N}$.

\hfill \break
\centering
\begin{tikzpicture}
    \tikzstyle{state}=[thick,draw=black,circle]
    \tikzstyle{decision}=[draw,shape=circle,fill=black]

    %\node[state] at (0,0) (s0) {$s_0$};
    %\node[state] at (2,0) (s0b) {};
    %\node[state] at (4,0) (s1) {$s_1$};
    %\node[state] at (4,-1) (s2) {$s_2$};

    %\draw (s0) edge [bend left, ->] node [midway, above] {} (s0b);
    %\draw[->] (s0b) -- (s1) node [midway, above] {0.5};
    %\draw (s0b) edge [bend left, ->] node [midway, below] {0.25} (s0);
    %\draw[->] (s0b) -- (s2) node [midway, below left] {0.25};
    \node[state] at (0,0) (s0) {$s_0$};
    \node[state] at (3,0) (s1) {$s_1$};
    \node[state] at (3,-1) (s2) {$s_2$};

    \draw[->] (s0) -- (s1) node [midway, above] {0.5};
    \draw (s0) edge [->, loop below] node [midway, below] {0.25} (s0);
    \draw[->] (s0) -- (s2) node [midway, below left] {0.25};

    \draw (s1) edge [->, loop right] node [midway, right] {1} (s1);
    \draw (s2) edge [->, loop right] node [midway, right] {1} (s2);
\end{tikzpicture}
\end{example}

\begin{example} The Drunkard's Walk is a well-known example of a Markov
    chain. One can imagine a drunk person starting in the middle of a road
    (state $s_0$) and then moving left or right at random. To understand
    Markov chains better the reader might want to calculate how many times
    will the drunk visit the middle of the road or how many steps will
    it take the drunk on average to reach a ditch ($s_{-2}, s_2$).

\hfill \break
\centering
\begin{tikzpicture}
    \tikzstyle{state}=[thick,draw=black,circle, minimum size=1.2cm]
    \tikzstyle{decision}=[draw,shape=circle,fill=black]

    \node[state] at (-5,0) (sm2) {$s_{-2}$};
    \node[state] at (-2.5,0) (sm1) {$s_{-1}$};
    \node[state] at (0,0) (s0) {$s_0$};
    \node[state] at (2.5,0) (s1) {$s_1$};
    \node[state] at (5,0) (s2) {$s_2$};

    \draw (sm2) edge [->, loop below] node [midway, below] {1} (sm2);
    \draw (sm1) edge [->, bend right] node [midway, below] {0.5} (sm2);

    \draw (s0) edge [->, bend right] node [midway, above] {0.5} (sm1);
    \draw (sm1) edge [->, bend right] node [midway, below] {0.5} (s0);

    \draw (s0) edge [->, bend right] node [midway, below] {0.5} (s1);
    \draw (s1) edge [->, bend right] node [midway, above] {0.5} (s0);

    \draw (s1) edge [->, bend right] node [midway, below] {0.5} (s2);
    \draw (s2) edge [->, loop below] node [midway, below] {1} (s2);
\end{tikzpicture}
\end{example}

\section{Markov Decision Processes}

Markov decision processes are similar to Markov chains, except now a
controller interacting with the process can in each state pick an action
and the next state is chosen according to a distribution on states
corresponding to this action.

\begin{definition}
A {\em Markov Decision Process} is a tuple $(S, s_0, A, E, \Delta)$, where
$S$ is a finite set of states,
$s_0 \in S$ is the initial state,
$A$ is a finite set of actions,
$E : S \to \powerset{A}$ gives the set of enabled actions in a state,
and $\Delta : S \times A \to \distribution{S}$ is a partial transition
function which assigns a probability distribution on states to an action
and a state.

It is assumed without loss of generality that for all $s \neq s'$ it
holds that $E(s) \cap E(s') = \emptyset$. If this did not hold the
actions could just be renamed.
\end{definition}

\begin{example}
    \label{ex_mdp}
    The MDP $(\{s_0, s_1, s_2\}, s_0, \{a, b\}, E, \Delta)$,
    where
    $E(s_0) =$ \linebreak $\{a,b\}$,
    $E(s_1) = \{c\}, E(s_2) = \{d\}$,
    and $\Delta(s_0, a) = \{(s_1, 1)\}$,
    $\Delta(s_0, b) = \{(s_1, 0.5),$ $(s_2,0.25), (s_0,0.25)\}$,
    $\Delta(s_1,c) = \{(s_1,1)\},
    \Delta(s_2,d) = $ \linebreak $ \{(s_2,1)\}$
    is depicted below.
    The edges labeled with letters denote the available actions
    and lead to smaller black dots, which mark the point of random
    choice of the successor state. With actions $c,d$ the black dots are
    omitted and the transition probability 1 is understood implicitly.

    The controller making decisions should choose action $a$ if
    they want to get to $s_1$, or (possibly repeatedly) choose $b$ if
    they want to get to $s_2$ (the achievement of this goal is not
    guaranteed).

\hfill \break
\centering
\begin{tikzpicture}
    \tikzstyle{state}=[thick,draw=black,circle]
    \tikzstyle{transition}=[draw,shape=circle,fill=black]

    \node[state] at (0,0) (s0) {$s_0$};
    \node[transition] at (2,1.5) (s0a) {};
    \node[transition] at (2,0) (s0b) {};
    \node[state] at (4,0) (s1) {$s_1$};
    \node[state] at (4,-1) (s2) {$s_2$};

    \draw[<-] (s0) -- node[above] {} ++(-1cm,0);

    \draw (s0) edge [bend left, ->] node [midway, above] {$a$} (s0a);
    \draw (s0a) edge [bend left, ->] node [midway, above] {$1$} (s1);
    \draw (s0) edge [bend left, ->] node [midway, above] {$b$} (s0b);
    \draw[->] (s0b) -- (s1) node [midway, above] {0.5};
    \draw (s0b) edge [bend left, ->] node [midway, below] {0.25} (s0);
    \draw[->] (s0b) -- (s2) node [midway, below left] {0.25};

    \draw (s1) edge [->, loop right] node [midway, right] {c} (s1);
    \draw (s2) edge [->, loop right] node [midway, right] {d} (s2);
\end{tikzpicture}

\end{example}

A standard use of Markov decision processes has been in areas where it
is useful to operate with rewards.

\begin{definition}
If $(S, s_0, A, E, \Delta)$ is a Markov decision process,
%$a \in A, s, s' \in S$, $a \in E(s)$, $\Delta(s,a)(s') > 0$,
and \linebreak $R : S \times A \times S \to \mathbb{R}$ is a function,
then $(S, s_0, A,E,\Delta,R)$ is a {\em Markov decision process with rewards}.
\end{definition}


\begin{definition}[Path]
    An {\em infinite path} is
    a sequence $\omega = s_0 a_0 s_1 a_1
    \ldots$ such that $a_i \in E(s_i)$ for all $i \in \mathbb{N}$.
    The set of all infinite paths of is denoted $IPaths$.

    A {\em finite path} is a prefix of an infinite path such that it
    ends with a state. The last state for a finite path $\rho$ is
    denoted $\last(\rho)$. The set of all finite paths is denoted
    $FPaths$.
\end{definition}


When following a path one might get stuck in an infinitely repeated
cycle of states and actions. The parts of MDP where this can happen are
called end components.  An example end component is shown in
\autoref{fig:non-triv-ec}.

\pagebreak

\begin{definition}[End component]
Let $\mathcal{M} = (S, s_0, A, E, \Delta)$ be an MDP,
let $S' \subseteq S$ and $A' \subseteq \bigcup_{s' \in S'} E(s')$.
The pair $(S', A')$ is an {\em end component},
if for every $s \in S', s' \in S, a \in A'$ it holds that
$$\Delta(s,a)(s') > 0 \implies s' \in S'$$
and there is a path between every two $s, s' \in S'$
using only actions $A'$.

An end component is {\em maximal} if it is maximal with respect to
the point-wise ordering of subsets.
\end{definition}

\begin{figure}[ht]
\begin{center}
\begin{tikzpicture}
    \tikzstyle{state}=[thick,draw=black,circle]
    \tikzstyle{transition}=[draw,shape=circle,fill=black]

    \draw[<-] (s0) -- node[above] {} ++(-1cm,0);

    \node[state] at (0,0) (s0) {$s_0$};
    \node[state] at (2,0) (s1) {$s_1$};
    \node[state] at (4,0) (s2) {$s_2$};

    \draw (s0) edge [bend left, ->] node [midway, above] {$a$} (s1);
    \draw (s1) edge [bend left, ->] node [midway, below] {$b$} (s0);
    \draw (s1) edge [->] node [midway, above] {$c$} (s2);

    \draw (s2) edge [->, loop right] node [midway, right] {d} (s2);
\end{tikzpicture}
\end{center}
\caption{An MDP with a non-trivial end component.}
\label{fig:non-triv-ec}
\end{figure}

The following definition introduces what is commonly called strategy,
policy, scheduler, controller or adversary. It is the decision maker in
the MDP which looks at the path traversed so far and assigns each action
$a$ in the last (current) state the probability of $a$ being chosen.

\begin{definition}[Strategy]
    Let $\mathcal{M} = (S,A,E,\Delta)$ be a Markov decision process
    and $\rho \in FPaths$.
    A {\em strategy} is a function
    $\sigma : FPath \to Dist(A)$
    such that\footnote{
The condition ensures that only actions which can
be chosen have non-zero probability assigned by the strategy.}
    $\sigma(\rho)(a) > 0 \implies a \in E(\last(\rho))$.

    A strategy $\sigma$ is {\em memoryless} if $\sigma(\rho)$ depends
    only on $\last(\rho)$. A strategy $\sigma$ is {\em deterministic}
    if $\sigma(\rho)(a) = 1$ for some $a \in E(\last(\rho))$.
\end{definition}

What happens once the strategy is fixed? The MDP becomes a Markov chain.
This can be seen in the examples, e.g. if the strategy is to always
choose $b$, the MDP from \autoref{ex_mdp} becomes the MC from
\autoref{ex_mc}.
Importantly when a strategy is fixed there is a probability
measure\footnote{While this may be intuitive it is not a trivial
statement, see Theorem 2.4 in \parencite{denumerable_mc} for a formal
treatment.} denoted $P^\sigma_{\mathcal{M},s}$ over the set of
$IPaths$ with an initial state $s$, which assigns to a set of paths the
probability they will be traversed.

%For very small MDPs this gives the first algorithm for evaluating
%their properties: for every strategy reduce the MDP to a Markov chain
%and evaluate the property using known Markov chain algorithms, then
%aggregate the results\footnote{The idea of this naive method can be
%significantly improved and instead of exploring a vast amount of
%strategies only a small portion of reasonably good strategies is
%explored \parencite{smc}.}.

\noindent {\bf Completeness of information about an MDP.}
One can distinguish between various levels of knowledge about an MDP,
usually, depending on the source of the model.
Complete information corresponds to knowing every part of the
definition of an MDP, as opposed to limited information where the MDP
can only be used as a black box from which can information be extracted
by sampling.
%For example $\Delta$ might
%not be known.
%he information about $\Delta$ may not be available (but
%reachability can still be effectively solved \parencite{atva14}).
This thesis is concerned only with MDPs with complete information.


\section{Verification}

{\em Formal verification} is the act of proving that a given system satisfies
a given property. {\em Model
checking} is an approach to formal verification which uses a model of
the system to verify the property.
In our case, the systems are modeled as Markov decision processes and
the properties are about the maximum probability of reaching a
state.

To simplify notation, in this section we often refer to a fixed MDP $\mathcal{M} =
(S,s_0,A,E,\Delta)$, and a set of target states $F$.

\noindent \textbf{Reachability probability.}
Let $\lozenge F$ be the set of all infinite paths that reach a state in $F$.
In this thesis we are concerned with maximising the reachability
probability $P^\sigma_{\mathcal{M},s_0}(\lozenge F)$ over the set of all
strategies $\sigma$.
We define the {\em value function} $V(s) = \sup_{\sigma}
P^\sigma_{\mathcal{M},s}(\lozenge F)$ for every state $s \in S$.

Importantly there is always a memoryless, deterministic strategy maximising the
reachability probability. This was proven by Puterman \parencite{puterman}
for reward maximization and the proof can be used with a simple
reduction for the case of verification \parencite{verification_complexity}.
For any MDP create an MDP with rewards, such that the reward function
is zero everywhere, except when entering a target state it gives reward 1
and then transitions into a new sink state (with reward 0).

In this section, we describe value iteration,
which is a standard algorithm for computing the maximum probability of
reaching a state in F. In the next section, a more involved algorithm
called BRTDP is described.

One approach not explained here is so-called {\em strategy iteration}
which starts with an arbitrary strategy and gradually improves it as
long as a change to the strategy is beneficial. Strategy iteration has
a clear stopping criterion as opposed to value iteration, however, in
practice, it does not perform better \parencite{forejt}.

Another approach we do not explain is formulating the problem as a system
of linear equations and solving it. The solution is precise and has
guaranteed correctness, but the computation quickly becomes expensive as
the model grows in size.  See \parencite{forejt} for a description of
this method.

\subsection*{Value Iteration}

Value iteration (in its variation for computing expected maximum rewards) is a
dynamic programming algorithm which was first described by Richard
Bellman\footnote{Known for introducing the term dynamic programming.}
\parencite{bellman} in 1957. We present its variation for computing the maximum
reachability probability.

Before showing the algorithm, we first note that it will need to
process all states. However, the values in some states are quite easy to
compute and a preprocessing will allow for reduction of the
state space.
These easy states are in set $Z \subseteq S$
of {\em zero states} or set $F' \subseteq S$ of {\em extended target states}.
States $z \in Z$ are such that $P^\sigma_{\mathcal{M},z}(\lozenge F) = 0$ holds,
and states $f \in F'$ are such that $P^\sigma_{\mathcal{M},f}(\lozenge F) = 1$
holds, both for any strategy $\sigma$.
Their computation is rather straightforward \parencite{forejt} (Section
4.1). Note that $F \subseteq F'$.

The main idea of value iteration is materialized in the following
recurrence relation for newly introduced variables $x_s^n, s \in S, n
\in \mathbb{N}$.
\[
x_s^n =
\begin{cases}
    1 & \text{if }s \in F' \\
    0 & \text{if }s \in Z \lor (s \not \in F' \land n = 0) \\
    \max\limits_{a \in E(s)} \sum\limits_{s' \in S} \Delta(s,a)(s') \cdot x_{s'}^{n-1}
    & \text{otherwise} % \text{if }s \in S \setminus (F \cup Z)
\end{cases}
\]
By computing $x^n$ for $n = 1,2,\ldots$ we gain increasingly precise
estimate of the actual maximum reachability probability,
formally $\lim_{n \to \infty} x^n_s = P_{\mathcal{M},s}(\lozenge F)$.
We refer to a proof of this statement in \parencite{puterman} with the
same reduction as in the case of existence of memoryless optimal
strategy.

This recurrence relation is now turned into a dynamic programming
algorithm as shown in \autoref{vi}. Instead of iterating $n$ times, the
algorithm proceeds with its computations while the convergence is not
slow (the threshold is given by some $\epsilon$).

\begin{algorithm}
\caption{Value Iteration}
\label{vi}
\begin{algorithmic}[1]
    \State $\forall s \in S,\; s \gets 1$ if $x \in F$ else $0$
    \Do
        \ForEach{$s \in S \setminus (F \cup Z)$}
            \State $x_s \coloneqq
            \max_{a \in E(s)} \sum_{s' \in S} \Delta(s,a)(s') \cdot x_{s'}$
        \EndFor
    \doWhile{the change of $x_s$ for any $s$ is greater than $\epsilon$}
\end{algorithmic}
\end{algorithm}

Value iteration is an easy method for computing the reachability
probability of a given MDP.  However, we only have a proof of convergence
and not a useful stopping criterion.  Furthermore, it is doing extra work
on models where only a small part needs to be explored to find a good
strategy as it has to compute its results for all the states.

The solution to the first issue is the {\em interval iteration}
algorithm \parencite{interval_iteration}, an algorithm in parts similar
to value iteration but which maintains lower and upper bounds of the
sought probability. The algorithm has a well-defined stopping criterion
and a bound on the running time.  We will explore solutions to the
second issue later with heuristic methods.

% http://www.sciencedirect.com/science/article/pii/S0304397516307095
% http://pageperso.lif.univ-mrs.fr/~benjamin.monmege/talks/MoVe2015.pdf

\section{Bounded Real-Time Dynamic Programming}
\label{sec:brtdp}

When only a portion of an MDP needs to be searched to find the right
strategy there is an opportunity to employ algorithms which avoid
searching the whole state space. Bounded Real-Time Dynamic Programming
(BRTDP) is such an algorithm. Originally developed for
the objective of finding the best-profit strategy
\parencite{profit_brtdp} the algorithm was adapted to the problem of
verification \parencite{atva14}.

As in the previous section, we refer to a fixed MDP $\mathcal{M} =$
\linebreak $(S,s_0,A,E,\Delta)$, and a set of target states
$F$. Further we fix $\epsilon$, an argument of the algorithm which sets
the required precision (maximum allowed distance of the result of the
algorithm from the correct value).

\pagebreak

Recall we have a unary value function defined
and for use in the algorithm let us define the binary {\em value function}
$V : S \times A \to [0,1]$ for all $s \in S$ and $a \in E(s)$ as follows
\[
    V(s,a) \coloneqq \sum_{s' \in S} \Delta(s,a)(s')V(s')
\]
This serves to represent the value in $s$ after taking action $a$.
BRTDP is learning $V$ by monotonically tightening its lower and upper
bounds $L, U : S \times A \to [0,1]$ during simulated runs of
$\mathcal{M}$ from the given initial state.
When $\max_{a \in E(s_0)} U(s_0, a) - \max_{a \in E(s_0)} L(s_0, a) < \epsilon$
the algorithm terminates.

We start with an important assumption, that
that $\mathcal{M}$ does not contain any end component
besides two trivial end components, one containing only the target state
1 with $F = \{1\}$, the other only the state 0 with $V(0) = 0$.

With this assumption BRTDP can be implemented as \autoref{alg:brtdp},
alternating between simulation and update phases until it has
sufficiently good knowledge about $V(s_0)$. In the simulation phase the,
algorithm samples a finite path from the initial state to one of states
$\{1, 0\}$, each time using an action maximising the upper bound.
In the update phase, the algorithm traverses the path backwards and
performs the Bellman update (known from value iteration),
using the best-known bounds, i.e.
$U(s) = \max_{a \in E(s)} U(s, a)$
and
$L(s) = \max_{a \in E(s)} L(s, a)$.
With the assumption of no non-trivial BRTDP converges almost surely to the correct value \parencite{atva14},
that is
the actual probability almost always lies between $L(s_0), U(s_0)$,
and the bounds converge such that they are at most $\epsilon$ (for a
given $\epsilon$) far from each other.


\begin{algorithm}
\caption{BRTDP for MDPs without end components}
\label{alg:brtdp}
\begin{algorithmic}[1]
\State $U(s,a) \gets 1, L(s,a) \gets 0 \; \forall s \in S, a \in E(s)$
\State $U(0,a) \gets 0, L(1,a) \gets 1 \; \forall a \in A$
\State $\omega \gets s_0$
\While{$U(s_0) - L(s_0) > \epsilon$ }
    % simulation
    \State \# Simulation Phase
    \While{$last(\omega) \not \in \{0,1\}$}
        \State $a \gets$ sample uniformly from
           $\argmax\limits_{a \in E(last(\omega))}
            U(last(\omega), a)$
        \State $s \gets$\footnote{The implementation of this line may
            vary, see Subsection ``Variants of BRTDP''}
            sample according to $\Delta(last(\omega), a)$
        \State $\omega \gets \omega \; a \; s$
    \EndWhile

    % update
    \State \# Update Phase
    \While{$\omega$ is not empty}
        \State $pop(\omega)$
        \State $a \gets pop(\omega)$
        \State $s \gets last(\omega)$
        \State $U(s,a) \coloneqq \sum_{s' \in S} \Delta(s,a)(s') U(s')$
        \State $L(s,a)\, \coloneqq \sum_{s' \in S} \Delta(s,a)(s') L(s')$
    \EndWhile
\EndWhile
\State \Return $(U(s_0), L(s_0))$
\end{algorithmic}
\end{algorithm}

\newpage

\subsection*{BRTDP for MDPs with End Components}
\autoref{alg:brtdp} is not guaranteed to converge
when the MDP contains non-trivial end components.

\begin{example}
    Below is an MDP with an end component
    $(\{s_0, s_1\}, \{a,c\})$. Let $F = \{s_2\}$.
    When BRTDP updates state $s_0$ or $s_1$ there is always a state (the
    other one in the EC) which has upper bound 1. This way the upper
    bound remains 1 after every iteration, even though the lower bound
    is correctly $0.5$. The algorithm does not converge for $\epsilon <
    0.5$.

\begin{center}
\begin{tikzpicture}
    \tikzstyle{state}=[thick,draw=black,circle]
    \tikzstyle{transition}=[draw,shape=circle,fill=black]

    \draw[<-] (s0) -- node[above] {} ++(-1cm,0);

    \node[state] at (0,0) (s0) {$s_0$};
    \node[state] at (2,0) (s1) {$s_1$};
    \node[transition] at (4,0) (s0b) {};
    \node[state] at (6, 0.7) (s2) {$s_2$};
    \node[state] at (6,-0.7) (s3) {$s_3$};

    \draw (s0) edge [bend left, ->] node [midway, above] {$a$} (s1);
    \draw (s1) edge [bend left, ->] node [midway, below] {$b$} (s0);
    \draw (s1) edge [->] node [midway, above] {$c$} (s0b);

    \draw[->] (s0b) -- (s2) node [midway, above] {0.5};
    \draw[->] (s0b) -- (s3) node [midway, below] {0.5};

    \draw (s2) edge [->, loop right] node [midway, right] {d} (s2);
    \draw (s3) edge [->, loop right] node [midway, right] {e} (s3);
\end{tikzpicture}
\end{center}
\end{example}

Fortunately, there is an on-the-fly method\footnote{The knowledge we have
about the MDP during computation suffices -- knowing the whole MDP is
not necessary.}
for resolving the problem in BRTDP
\parencite{atva14}, which we present to an extent important for our
future use of BRTDP.

During the simulation phase, the algorithm periodically (every $k_i$
steps) creates an auxiliary MDP based on the states visited so far and
their neighbours. The algorithm then identifies maximal ECs in this auxiliary
MDP.

If an end component is found it is collapsed, i.e. the states are merged into a
single state and functions $E, \Delta$ are naturally transformed to work
the same way in the modified MDP.

Finally, the merged state has its $U, L$ updated for every action in the
end component. If there was a target state in the end component then the
merged state is marked as a target state. If there was no target state and
there is no outgoing action from the merged state then it is marked as a
zero state.

\pagebreak

\subsection*{Variants of BRTDP}
When BRTDP is in state $s$ and it has chosen action $a$,
the choice of the next state does not necessarily have to be done by
sampling the transition distribution $\Delta(s,a)$ (we call this variant
HIGH-PROB) but can be instead chosen with probability
$\Delta(s,a)(s') \cdot (U(s') - L(s'))$, we call this variant MAX-DIFF.
One can think of other variants, for example round-robin choice.

\begin{example}
\label{brtdp_adversary}
We conclude with an example MDP, which is hard for most variants of
BRTDP when the target is the final state.
For example the HIGH-PROB variant has probability $0.99$ of returning to
the initial state from every state until reaching the final state.

\begin{center}
\begin{tikzpicture}
    \tikzstyle{state}=[thick,draw=black,circle]
    \tikzstyle{transition}=[draw,shape=circle,fill=black]
    \tikzstyle{loop}=[looseness=5, in=120, out=60]

    \node[state] at (0,0) (s0) {0};
    \node[transition] at (1.6,0) (s0b) {};
    \node[state] at (3.2,0) (s1) {1};
    \node[transition] at (4.8,0) (s1b) {};
    \node[state] at (6.4,0) (s2) {2};
    \node[transition] at (8,0) (s2b) {};
    \node[state] at (9.6,0) (s3) {3};

    \draw (s0) edge [bend left, ->] node [midway, above] {$a$} (s0b);
    \draw[->] (s0b) -- (s1) node [midway, above] {0.01};
    \draw (s0b) edge [bend left, ->] node [midway, above] {0.99} (s0);

    \draw (s1) edge [bend left, ->] node [midway, above] {$b$} (s1b);
    \draw[->] (s1b) -- (s2) node [midway, above] {0.01};
    \draw (s1b) edge [bend left, ->] node [midway, above] {0.99} (s0);

    \draw (s2) edge [bend left, ->] node [midway, above] {$c$} (s2b);
    \draw[->] (s2b) -- (s3) node [midway, above] {0.01};
    \draw (s2b) edge [bend left, ->] node [midway, above] {0.99} (s0);

    \draw (s3) edge [loop,->] node [above] {$d$} (s3);
\end{tikzpicture}
\end{center}
\end{example}

\chapter{Monte Carlo Tree Search}
\label{ch_mcts}

Monte Carlo methods\footnote{Not to be confused with
Monte Carlo algorithms which are precisely defined as the algorithms
solving the decision problems in classes BPP and RP.} are using random sampling to estimate the correct
solution to a problem. The first serious use of Monte Carlo methods was
by Stanislaw Ulam and John von Neumann during their work on the
Manhattan project, but the technique has since spread into many areas of
science due to its general applicability.

One of the celebrated Monte Carlo methods
is the simulated annealing algorithm (so called due to its
origin in statistical physics) which is an improved version of
Metropolis algorithm (invented by a Manhattan project scientist
Nicholas C. Metropolis and others).

This method found its way into game
theory in 1993 when it was applied to the board game Go
\parencite{MonteCarloGo}. The approach was later further improved
\parencite{MonteCarloGoDevel} but the real breakthrough came in 2006
when Coulom \parencite{Coulom} and Kocsis, Szepesvári \parencite{Kocsis}
independetly explored the idea of maintaing a tree which would guide the
search for strategies -- thus discovering Monte Carlo Tree Search.
This was eventually used in the AlphaGo program
\parencite{alphago}, the first
computer program to beat professional human players.

Monte Carlo Tree Search (MCTS) is, in short, a
heuristic search algorithm for finding good strategies in complex
decision processes by combining standard approaches of artificial
intelligence and computational statistics: tree search and sampling.

In this chapter the general MCTS scheme is defined and a concrete instance
called UCT is shown together with its application to maximizing rewards
in MDPs and games. The chapter is based mostly on a thorough MCTS survey
paper \parencite{mcts_survey}.

Since the research into MCTS focuses mainly on using MCTS for reward maximization
we also focus on that in this chapter unlike the rest of the thesis.

TODO: define MDP with rewards here (but mention it in the mdp chapter
too, just briefly, informaly)

\section{General MCTS Scheme}

MCTS iteratively builds a tree which approximates possible resulting
rewards of strategies in the decision process. In each iteration the
tree guides the search to balance between exploitation of known good
strategies and exploration of new strategies. When the search leaves the
tree it proceeds at random and upon terminating it adds a new leaf to
the tree and updates its ancestors with the result. This is summarized
in \autoref{mcts}.


\begin{algorithm}
\caption{General Monte Carlo Tree Search method}
\label{mcts}
\begin{algorithmic}
\Function{MCTS}{$s_0$}
    \State Let $v_0$ be the root of the MCTS tree, with $v_0.state = s_0$.
    \While{within computational budget}
        \State $v_l \gets \Call{TreePolicy}{v_0}$
        \State $\Delta \gets \Call{DefaultPolicy}{v_l}$
        \State $\Call{Backup}{v_l, \Delta}$
    \EndWhile
    \State \Return Action from $v_0$ to the best node (by some
    given metric).
\EndFunction
\end{algorithmic}
\end{algorithm}

\section{Upper Confidence Bound for Trees}

The most common implementation of the general scheme is {\em Upper
Confidence Bound for Trees} (UCT) which utilizes formula \ref{UCB}.
In this formula $\overline{X}_i$ represents the expected outcome
from node $i$, $n$ is the number of visits to all nodes, $n_i$ is the
number of visits to node $i$. $C$ is an arbitrary constant.

\begin{equation}
\label{UCB}
UCT_i = \overline{X}_i + C \sqrt{ \frac{2 \ln n}{n_i} }
\end{equation}

Selecting a tree node which maximizes this value makes the algorithm balance
between exploitation of known good strategies and exploration of new.
Constant $C$ instructs the algorithm how much weight to give to
exploration. TODO: Mention the result of Kocsis and Sepe... which proves
$1/\sqrt(2)$ to be optimal under some circumstances.

\autoref{uct} is implementation of the general \autoref{mcts}. As UCT is
the most common type of MCTS algorithm, the terms are often used
interchangeably in literature.

TODO: Define precisely the problem uct is solving (it should be
maximization of rewards in MDP).

\begin{algorithm}
    \caption{Upper Confidence Bound for Trees}
\label{uct}
\begin{algorithmic}
\Function{UCT}{$s_0$}
    \State TODO
    \State Let $v_0$ be the root of the MCTS tree, with $v_0.state = s_0$.
    \While{within computational budget}
        \State $v_l \gets \Call{TreePolicy}{v_0}$
        \State $\Delta \gets \Call{DefaultPolicy}{v_l}$
        \State $\Call{Backup}{v_l, \Delta}$
    \EndWhile
    \State \Return Action from $v_0$ to the best node (by some
    given metric).
\EndFunction
\end{algorithmic}
\end{algorithm}

\section{Solving Games}

In this section a brief introduction to game theory is given, starting
with definitions of games, strategies and solutions to games. We
proceed by presenting the standard {\em minimax} algorithm, then showing
how MCTS can be used to solve games and how it compares with minimax.
Lastly we show how MCTS is used to play the game of Go, which has been a
fruitful subject of research.
Observing where it performs good and where it does not provides insight
into MCTS and is a useful starting point for understanding the results
of evaluation in \autoref{ch_evaluation}.


Our definition of a game is that of a {\em perfect-information
extensive-form game}. The perfect-information in games corresponds to
full observability in MDP.  The reader can notice other similarities
with MDPs as well, for example games have states, final states, actions
and enabled actions, as well as rewards. On the other hand the
transition function is deterministic.

\begin{definition}
    A {\em game} is a tuple $G = (N, S, F, A, E, \Delta, \rho, u)$,
    where
    \begin{itemize}
        \item $N \subseteq \mathbb{N}$ is a finite set of players,
        \item $S$ is a set of states,
        \item $F \subseteq S$ is a set of final (terminal) states,
        \item $A$ is the set of actions,
        \item $E : S \to \mathcal{P}^+(A)$ is a function which tells the player
            which non-empty set of actions can be played in a given state,
        \item $\Delta : S \times A \to A$ is the transition function,
        \item $\rho : S \to N$ is a function which determines who plays in each state,
            and finally
        \item $u = (u_1,\ldots,u_n)$ is the tuple of
            utility (reward) functions where each $u_i, i \in N$ is a function
            $u_i : F \to \mathbb{R}$.
    \end{itemize}
\end{definition}

TODO: there may be some more conditions I didn't realize in the definitions, check
ia168

Game starts in a distinguished state $s_0$ and players take turns until
a terminal state (in $F$) reached. In a turn a player in state $s$ chooses
action $a$ from $E(s)$ and the action leads to state $\Delta(s,a)$.
If the state $\Delta(s,a)$ is terminal, then every player $i$ receives
the reward $u_i(\Delta(s,a))$.

A common choice of the utility function is $+1, 0, -1$ for victory, draw
and loss, respectively.

\begin{example}
\end{example}

\begin{definition}
    A {\em strategy} of player $i$ in a game $G$ is a function
    $\pi_i : S \to \distribution{A}$, such that for all $s \in S$ and $a
    \in A$ it holds that if $\pi_i(s)(a) > 0$ then $a \in E(s)$.

    A {\em strategy profile} is an $N$-tuple with a strategy for each player.
\end{definition}

We assume the players are rational and thus pick strategies which
optimize for their goals. A goal may be to maximize the player's reward,
another goal may be to get a higher score than the opponents.

\begin{example}
    \label{example_goals}
    In the game below let Alice be player 1 and Bob player 2.
    There is a single turn made by Alice in the black node.
If her goal is to beat Bob she will pick the left node.
If her goal is to maximize her profit she will choose the right one.

    \begin{center}
\begin{tikzpicture}
\tikzset{
solid node/.style={circle,draw,inner sep=1.5,fill=black},
hollow node/.style={circle,draw,inner sep=1.5}
}

\node(0)[solid node]{}
    child{node[hollow node,label=below:{$(2,1)$}]{}}
    child{node[hollow node,label=below:{(5,10)}]{}};
\end{tikzpicture}
\end{center}
\end{example}

For simplicity we restrict
ourselves to two player games in the following text unless otherwise
noted.


\subsection{Zero-sum games and minimax}

In zero-sum games one player wins at the cost of the other losing.
There are plenty of examples of such games, e.g. Chess, Go, which makes
them an important subject of study.

\begin{definition}
    Let $G = (N, S, F, A, E, \Delta, \rho, u)$ be a game.
    $G$ is said to be {\em zero-sum} if the rewards always sum to zero,
    that is $\sum_{i \in N} u_i(s) = 0$ for all $s \in F$.
\end{definition}

An important type of strategy is {\em minimax}. A player playing this
strategy is minimizing their potential maximum loss\footnote{Some might
prefer calling it {\em maximin} for maximization of the minimum reward,
which is equivalent to the first definition in zero-sum games.}. If both players play a
minimax strategy, the strategy profile is a Nash
equilibrium\footnote{
A strategic profile is a Nash equilibrium, if no player can get a higher
reward by switching a strategy.
    }$^,$\footnote{The {\em minimax theorem} was proven by John
von Neumann.}.

The {\em minimax algorithm} computes the value of each leaf in a game
tree assuming the opponent tries to harms the player as much as
possible. The algorithm then tells the player to play the action going
into the subtree with the highest value node.

\autoref{negamax}, the {\em negamax algorithm}, is a minor simplification
of minimax, equivalent to minimax in zero-sum games.

\begin{algorithm}
\caption{Negamax}
\label{negamax}
\begin{algorithmic}
\Function{negamax}{$node$}
    \If{$node$ is a leaf}
        \State \Return hmm.. how to write it
    \EndIf
    \State \Return $\max \{ - \Call{negamax}{child}
        \mid child \text{ of } node \}$
\EndFunction
\end{algorithmic}
\end{algorithm}

TODO: example execution

For larger games a variation called {\em minimax search} is used, which
performs the computation only to a limited depth, and then uses
a heurisic to evaluate the last node it processes (unless it is a leaf).

\subsection{Solving with MCTS}

From the negamax algorithm in the previous section there is an easy step
to solving games with MCTS. \autoref{mcts_negamax} is an
implementation of the Backup function in UCT for finding a good
action a player should take in a zero-sum game.

\begin{algorithm}
\caption{Negamax MCTS Backup}
\label{mcts_negamax}
\begin{algorithmic}
\Function{backup}{$v, \Delta$}
    \While{$v$ is not $null$}
        \State $N(v) \gets N(v) + 1$
        \State $Q(v) \gets Q(v) + \Delta$
        \State $\Delta \gets - \Delta$
        \State $v \gets $ parent of $v$
    \EndWhile
\EndFunction
\end{algorithmic}
\end{algorithm}

TODO: Example execution?

An important theoretical result is
that MCTS converges to minimax \parencite{Kocsis},
so eventually, after a lot of MCTS iterations, there is a negligible
difference between the results of the two methods and they pick the same
next move. However in
applications we want to know when to use minimax (\autoref{negamax}) and
when to use UCT (\autoref{uct} with \autoref{mcts_negamax}) without
iterating UCT for a long time.

While UCT achieved success in many games \parencite{mcts_survey},
for example in chess it does not perform well. The reason being that chess
games contain a lot of trap states (states from which the opponent
has a guaranteed victory) which are reachable just in a few turns. UCT
then spends a lot of time exploring these fruitless parts of the search
tree \parencite{mcts_vs_chess}.

\subsection{Computer Go}

Go is an example of a game where MCTS has proven to be a good choice,
most recently with the AlphaGo program \parencite{alphago}.
We shortly introduce Go and how MCTS is used to
play it.

The basic rules of Go are simple. The
standard board is a $9\times9$ (for beginners) or $19 \times 19$ grid.
Two players, black and white, alternate in their moves, each
placing a single stone of their color on an intersection of lines.
By surrounding the stones of the opponent a player captures all the
surrounded stones. The game ends when both players agree to end and the
winner is the player with greater sum of controlled territory and
captured stones.
The rules are well explained in detail at
\href{http://playgo.to/iwtg/en/}{http://playgo.to/iwtg/en/}.


TODO: How is MCTS used in some simple Go program.
Link to AlphaGo for advanced technique.

A Go player can observe that unlike in Chess there is seldom a quick way
to lose, as a loss of a small part of territory may be reversed, however
losing an important piece is hardly reversible. It seems plausible this
generalizes to territory versus piece based games.

We end our detour to games with Arimaa, a game specifically designed in
2002 to be easy for humans but hard for computers, for example by
allowing more moves per turn. Eventually Arimaa players lost to a
program in 2015 \parencite{arimaa}. The program uses techniques similar
to chess programs like alpha-beta pruning and on top of that employs
heuristics inspired by the best human players. See \parencite{jakl} for
MCTS related insights into Arimaa.

\chapter{MCTS in MDP Verification}

In this chapter three new algorithms for verification of Markov
decision processes are presented. One called MCTS-BRTDP
is a fusion of MCTS with BRTDP, the second called Bounded MCTS (BMCTS) is more
similar to the general MCTS scheme but still maintains bounds. The third one called
BRTDP-UCB uses the UCB formula to select the next action in BRTDP.

\section{MCTS-BRTDP}

MCTS-BRTDP is a variant of MCTS where the tree policy is the standard
selection of a node with the maximum UCB value. The default policy is
BRTDP as described in \autoref{ch_mdp} and the bound updates then continue
through the tree upwards to its root.

\begin{algorithm}
\caption{MCTS-BRTDP}
\label{mcts-brtdp}
\begin{algorithmic}
\Function{MCTS-BRTDP}{$s_0$}
    \State Let $v_0$ be the root of the MCTS tree, with $v_0.state = s_0$.
    \While{$U(s_0) - L(s_0) > \epsilon$}
        \State $v_l \gets \Call{TreePolicy}{v_0}$
        \State $\Delta \gets \Call{BRTDP}{v_l}$
        \State $\Call{Backup}{v_l, \Delta}$
    \EndWhile
    \State \Return Action from $v_0$ to the best node (by some
    given metric).
\EndFunction

\Function{TreePolicy}{$s$}
\Repeat
\Until{$s \neq s_0$}
\EndFunction

\Function{Backup}{$s_0, s$}
\Repeat
    \State $U(s,a) \coloneqq \sum_{s' \in S} \Delta(s,a)(s')U(s')$
    \State $L(s,a)\, \coloneqq \sum_{s' \in S} \Delta(s,a)(s')L(s')$
    \State $s \gets parent(s)$
\Until{$s \neq s_0$}
\EndFunction
\end{algorithmic}
\end{algorithm}

TODO: Prove PAC.

\section{Bounded MCTS}

BMCTS is an algorithm similar to MCTS-BRTDP but the default policy
selects the next node (TODO: action?) at random with uniform
distribution.

TODO: Describe the advantage and disadvantages.

The functions \textsc{TreePolicy} and \textsc{Backup} are implemented in
the same way as in \autoref{mcts-brtdp}.

\begin{algorithm}
\caption{BMCTS}
\label{bmcts}
\begin{algorithmic}
\Function{BMCTS}{$s_0$}
    \State Let $v_0$ be the root of the MCTS tree, with $v_0.state = s_0$.
    \While{$U(s_0) - L(s_0) > \epsilon$}
        \State $v_l \gets \Call{TreePolicy}{v_0}$
        \State $\Delta \gets \Call{DefaultPolicy}{v_l}$
        \State $\Call{Backup}{v_l, \Delta}$
    \EndWhile
    \State \Return Action from $v_0$ to the best node (by some
    given metric).
\EndFunction

\Function{DefaultPolicy}{$s$}
\EndFunction

\end{algorithmic}
\end{algorithm}


TODO: Prove PAC.


\section{UCB in BRTDP}

We further present one algorithm which is not based on MCTS but
incorporates the $UCB$ exploration term into BRTDP. TODO: pseudocode?
probably not, it is just BRTDP where the action is chosen according to
UCB, is it PAC? can we prove it?

\chapter{Evaluation}
\label{ch_evaluation}

This chapter explains the basics of working with PRISM model checker,
then describes how are the algorithms implemented as a part of PRISM,
how they behave on small models, and how do they compare to other
methods on standard models.

\section{PRISM, Probabilistic Model Checker}

PRISM \parencite{prism}
({\em probabilistic model checker}) is a program/framework
for formal modelling and analysis of probabilistic systems.
We show how to use PRISM to describe MDPs, their properties,
and how to check them.

\subsection*{Describing MDPs with the PRISM language}
PRISM has a language for description of Markov decision processes
based on the formalism of Alur and Henzinger \parencite{ReactiveModules}.
A brief example is given below for the readers who wish to try our
algorithms on small models which they can describe and solve by hand.
The the definitive guide to the language is available online on the
PRISM homepage \parencite{prism_lang}.

The PRISM language describes {\em modules} (interacting actors),
their states (using variables) and transitions between the states.
An example single PRISM language line is below. The line translates as:
if condition \verb|guard| is satisfied the actor can choose action \verb|act|
and
with probability \verb|prob_1| update \verb|update_1| will happen,
with probability \verb|prob_2| update \verb|update_2| will happen,
and so on.

\begin{verbatim}
[act] guard -> prob_1 : update_1 + prob_2 : update_2 + ...
\end{verbatim}

An example module is described below and is equivalent to the MDP
depicted in \autoref{moduleM}.

\smallskip
\begin{verbatim}
mdp // Tell PRISM this file describes an MDP
module M
    s : [0 .. 3] init 0;
    [a] s=0 -> (s'=0);
    [b] s=0 -> 0.9:(s'=1) + 0.1:(s'=0);
    [c] s=1 -> 0.9:(s'=2) + 0.1:(s'=1);
    [d] s=2 -> (s'=2);
endmodule
\end{verbatim}
\smallskip

\begin{figure}
\begin{tikzpicture}
    \tikzstyle{state}=[thick,draw=black,circle]
    \tikzstyle{transition}=[draw,shape=circle,fill=black]
    \tikzstyle{loop}=[looseness=5, in=120, out=60]

    \node[state] at (0,0) (s0) {$s = 0$};
    \node[transition] at (2,0) (s0b) {};
    \node[state] at (4,0) (s1) {$s = 1$};
    \node[transition] at (6,0) (s1b) {};
    \node[state] at (8,0) (s2) {$s = 2$};

    \draw (s0) edge [loop,->] node [above] {$a$} (s0);

    \draw (s0) edge [bend left, ->] node [midway, above] {$b$} (s0b);
    \draw[->] (s0b) -- (s1) node [midway, above] {0.9};
    \draw (s0b) edge [bend left, ->] node [midway, below] {0.1} (s0);

    \draw (s1) edge [bend left, ->] node [midway, above] {$c$} (s1b);
    \draw[->] (s1b) -- (s2) node [midway, above] {0.9};
    \draw (s1b) edge [bend left, ->] node [midway, below] {0.1} (s1);

    \draw (s2) edge [loop,->] node [above] {$d$} (s2);
\end{tikzpicture}
\caption{Module M}
\label{moduleM}
\end{figure}

A property we might be interested in is the maximum probability of
eventually reaching state \verb|s=2|, this could be described to PRISM
with the string \verb|Pmax=? [F s=2]|, where $F$ represents the
$\lozenge$ symbol used previously. A user can use standard logical
connectives for describing the target states. At the moment our
implementation does not support timed properties (i.e., \verb|F <= x| is not
supported), however, we expect it is not too hard to implement such
functionality.

\subsection*{Running PRISM}

We have a model description and a property but before we analyze it
PRISM has to be installed.
There is a modified version of PRISM distributed with this thesis. It
can be built by issuing the \verb|make| command inside the
\verb|prism/prism| directory. Java is a required prerequisite.
Once built the PRISM binary is available at \verb|prism/prism/bin/prism|.

Now use PRISM to analyze the model -- its name is passed as the first
argument and \verb|-pf| specifies the property we want to check.
The \verb|-ex| flag will be described later.
Running the command below
confirms our expectations: the maximum probability of eventually
reaching state \verb|s=2| is 1. If not specificed otherwise, the
algorithm used is value iteration.

\medskip
\begin{verbatim}
./prism modelM.nm -pf 'Pmax=? [F s=2]' -ex
\end{verbatim}
\medskip

The following tells PRISM to use MCTS-BRTDP where the next state in
BRTDP is chosen to be the one with highest upper bound.

\medskip
\begin{verbatim}
./prism modelM.nm -pf 'Pmax=? [F s=2]' -heuristic_verbose \
 -heuristic MCTS_BRTDP -next_state HIGH_PROB
\end{verbatim}
\medskip

The UCB constant can be chosen with
\verb|-ucb1constant|. If not provided the value is set to
$1/\sqrt{2}$.

The heuristic method can be chosen out of the following:\linebreak
\verb|MCTS_BRTDP|, \verb|BRTDP|, \verb|BRTDP_UCB|. BMCTS is chosen by
using method \verb|MCTS_BRTDP| together with \verb|-next_action 5|.
The next state heuristics we use are
\verb|HIGH_PROB|, \verb|MAX_DIFF|.
The variation of BRTDP using UCB to select the next action is chosen by
adding \verb|-next_action 2|.

\subsection*{Data structures}
The \verb|-ex| switch used in the value iteration example above tells
PRISM to use the explicit computation engine.
The explicit computation engine explores the MDP and stores it in
a sparse matrix before the value iteration algorithm is run.
PRISM also offers three symbolic computation engines based on binary decision
diagrams.

All the heuristic methods use a data structure implemented in
\verb|prism/prism/src/heuristic/CachedModelGenerator.java|.
With this {\em model generator} data structure PRISM will not build the
whole MDP from its description unless asked to.
Asking the model generator to reveal parts of the model results in the
construction of an {\em explicit model} which is cached in the memory.
Such construction is computationally expensive and often unnecessary
which is when the heuristic methods perform so well.

\section{Implementation}

We describe how the pseudocode described in \autoref{ch_mcts} maps to
the implementation in PRISM. The implementation can be found in
\verb|prism/prism/src/heuristics|.

To represent the MCTS tree we use classes \verb|MCTree.java| and
\linebreak
\verb|MCNode.java|. The tree class has an important method \verb|unfold|
which asks the model generator to add the next states of a given state
to the explicit model and adds them to the tree.

The next state heuristics are implemented in a straightforward way
inside directory \verb|nextstate|. The UCB heuristic is implemented in
directory \verb|treeheuristic|.

MCTS-BRTDP is implemented in \verb|search/MctsBrtdp.java|. The entry
point is \verb|computeProb|, subsequently
\verb|monteCarloTreeSearch| is invoked until the stopping condition
is reached (see method \verb|isDone|). Method
\verb|monteCarloTreeSearch| first selects and expands a tree state
(using \verb|treeSelectAndExpand|),
then uses \verb|exploreAndUpdate| implemented in BRTDP
(\verb|search/HeuristicBrtdp.java|) as the rollout and propagates
the values using updates to the root.

As we have observed problems only once during thousands of runs on our
models, we have not implemented the removal of subtrees induced by nodes
corresponding to a state which is contained in a collapsed MEC.

BMCTS is implemented by modifying MCTS-BRTDP. The modification is turned
on when the flag \verb|next_action 5| is added to a command using
MCTS-BRTDP. This change makes the BRTDP implementation chose next action
uniformly at random instead of the BRTDP-way by upper bound.

\section{Behaviour on Small Models}

For small models we used visualization to observe how MCTS-BRTDP solves
them. To render the progress of MCTS-BRTDP into a series of pictures,
the last lines of \verb|MctsBrtdp.monteCarloTreeSearch| has to be uncommented.
Due to the randomized nature of the algortihms a researcher should
observe more runs before making conclusions.

Three simple models are presented with a description of the methods'
approach to solving them.
The first is a model resembling a binary tree, second is
the BRTDP adversary, the third model offers a choice between a simple
path and a complex cloud. They can be found in directory
\verb|small_models| attached to the thesis.

\subsection*{Binary Tree Model}

The ``binary tree'' model features two decisions ({\em left} and {\em
right}) in each state and each decision has two successors,
the {\em left} one occurs with probability 0.2,
the {\em right} one with 0.8. A path through the model goes through 4 states
before it reaches a ``leaf'' state. Every leaf state of the subtree
induced by the left decision in root leads to state 85, every leaf state
of the other subtree leads to state 86. We ask what is the maximum
probability of reaching state 85.

By running the program repeatedly it can be observed that MCTS-BRTDP
almost evenly explores the branches in a balanced way while utilizing
BRTDP rollouts to search for promising paths. BRTDP on its own
selects a branch it knows the least about until it learns the upper bound
for the right part of the tree is 0 and therefore it has to focus on the
left part. The VCB for MCTS-BRTDP focuses on the left part quickly due
to successful results, however it still explores the right part a bit to
look for possibly missed target states and to decrease the upper bound.

\subsection*{BRTDP Adversary}

In \autoref{brtdp_adversary} MCTS-BRTDP has a clear advantage as it
traverses to a leaf of the tree and then adds a node to it with each iteration.
Soon the tree reaches the target state and it remains to perform updates
equivalent to value iteration.


\subsection*{Cloud And Path Model}

The model has an initial state,
``left part'', and ``right part''. Left part is a simple path to a
target state. Right part is comprised of states with randomly selected
choices and transitions but without a target state.
From the initial state there is an action to go left with probability
0.8 and right with probability 0.2, and another action with the same
effect but reversed probabilities.

MAX-DIFF BRTDP quickly learns that the upper bound in right part is
zero, abandonds this part of the MDP to focus on the left part and learn
its upper and lower bound 0.8.

MCTS-BRTDP is forced to explore the right part too but learns the
correct value soon. On the other hand MCTS-BRTDP with the VCB formula
finds the target in the left part and then focuses too much on this part
even though what it needs is to learn that the right part has upper
bound 0 -- this variant is thus taking very long to find the correct
probability unless it is configured with a high multiplier in the
exploration term of the VCB formula.

Observing the VCB variant of MCTS-BRTDP on this and the binary tree
models suggests that it is quick to find good lower bounds. However,
then it commits to the good paths of the MDP so much that it has trouble
computing a better upper bound in the parts which do not lead to a
target way (as quickly as the exploited parts).

\section{PRISM Benchmark Suite and Other Models}

Experimental evaluation was done mainly on standard models
distributed with PRISM. Most of the MDPs described by these models are
concerned with network configuration where randomness plays important
role in achieving a common goal. We provide a brief, incomplete
description of the {\em zeroconf} model to offer basic familiarity and
refer the reader to thorough descriptions of all the models on PRISM's
website
\href{http://www.prismmodelchecker.org/casestudies/}{www.prismmodelchecker.org/casestudies/}.

We also created new MDPs by combining the PRISM models with the MDP
which is hard for BRTDP (\autoref{brtdp_adversary}). We describe them in
the last subsection.

The description of all the models in the PRISM language can be found inside
\verb|tests/reachability/models| in the source codes attached to this
thesis.

\subsection*{Zero-configuration networking}

{\em
Zeroconf}\footnote{\href{http://www.prismmodelchecker.org/casestudies/zeroconf.php}{http://www.prismmodelchecker.org/casestudies/zeroconf.php}} model corresponds to a set of computers establishing a
network without a given leader. Such leader could be a human setting
static IP addresses or a DHCP server which would need to be configured
up front -- in both cases there is extra work required.

The zero-configuration networking protocol describes how should the
computers proceed. Upon connecting to the network a computer picks an IP
address at random and broadcasts its choice via an ARP packet called
{\em probe} repeated $K$ times ($K=4$ by the standard) with two second
delay. If another computer responds to one these
ARP packets the original sender will then pick another IP address and
repeat the process. If it does not receive a response it then broadcasts
twice an ARP packet asserting this computer's use of the chosen IP
address.

%\subsection*{IEEE 1394 FireWire}

%FireWire is an interface standard for serial bus

%\subsection*{Wireless LAN}

%\subsection*{Randomised Consensus Shared Coin Protocol}

%\subsection*{mer}

\subsection*{Combining PRISM models with BRTDP adversary}

We combined the MDPs in two ways. The first is by {\em branching} and the
resulting shape is shown in \autoref{fig:branching}.

\begin{figure}[h]
\begin{center}
\begin{tikzpicture}
    \tikzstyle{state}=[thick,draw=black,circle]
    \tikzstyle{transition}=[draw,shape=circle,fill=black]
    \tikzstyle{loop}=[looseness=5, in=120, out=60]

    \node[state] at (0,0) (s0) {0};
    \node[transition] at (1.6,0) (s0b) {};
    \node[state] at (3.2,0) (s1) {1};
    \node[transition] at (4.8,0) (s1b) {};
    \node[state] at (6.4,0) (s2) {2};
    \node[transition] at (8,0) (s2b) {};
    \node[state] at (9.6,0) (s3) {3};

    \node [cloud, draw,cloud puffs=10,cloud puff arc=120, aspect=2,
        inner ysep=1em] at (2,-3) (zconf) {zeroconf};

    \draw (s0) edge [bend right, ->] node [midway, right] {$b$} (zconf);

    \draw (s0) edge [bend left, ->] node [midway, above] {$a$} (s0b);
    \draw[->] (s0b) -- (s1) node [midway, above] {0.01};
    \draw (s0b) edge [bend left, ->] node [midway, above] {0.99} (s0);

    \draw (s1) edge [bend left, ->] node [midway, above] {} (s1b);
    \draw[->] (s1b) -- (s2) node [midway, above] {0.01};
    \draw (s1b) edge [bend left, ->] node [midway, above] {0.99} (s0);

    \draw (s2) edge [bend left, ->] node [midway, above] {} (s2b);
    \draw[->] (s2b) -- (s3) node [midway, above] {0.01};
    \draw (s2b) edge [bend left, ->] node [midway, above] {0.99} (s0);

    \draw (s3) edge [loop,->] node [above] {} (s3);

\end{tikzpicture}
\end{center}
    \caption{Combining zeroconf model with the BRTDP adversary in a
    branching manner.}
    \label{fig:branching}
\end{figure}

The second way to combine MDPs is parallel composition, which is done by
using modules in the PRISM language. Each state of the composed MDP is a member
of the product set of the sets of states of each of the modules. The
choice of the next transition is non-deterministic, i.e. a strategy does
not decide only which transition in a module to use but also which
module to use (the strategy for the MDP can be viewed as a member of the
product set of the set of strategies for each of the modules).

\section{Experimental Comparison}

On the following pages we present the results of our measurements.
In the presented tables rows correspond to models and their
configuration, columns to methods and chosen tree heuristic constant (in
parenthesis). Each
table cell contains comma separated running time in seconds and number of visited
states of the MDP\footnote{The number of states is, unlike running time,
agnostic to implementation details.}\textsuperscript{,}\footnote{Value iteration has to construct the
whole model in memory but we show the number of states after VI
preprocessing.}. Each experiment shown in the tables below was repeated
at least five times and the results were averaged\footnote{There was no
significant variance in the measured times so averaging provides
representative results.}.

We used a machine equipped with
40 cores of \verb|Intel(R) Xeon(R)| \verb|CPU E5-2630 v4 @ 2.20GHz| processors,
OpenJDK Runtime Environment 1.8 and JVM configured to use 4 GB of memory
and 512 MB stack size.

Our benchmarking script which allows to describe each measurement
in an easy, declarative way is described in the appendix. See
\verb|prism/prism/tests/reachability/scripts/run_benchmark.py| for
\linebreak an example.
Results of many measurements, some of which were selected for the tables
presented later, can be found in directory \linebreak \verb|benchmarks| together
with a script for transforming them into a \LaTeX\ table.

In \autoref{table:general_comparison} we can see that MCTS-BRTDP-MAXDIFF can
keep up with BRTDP-MAXDIFF on the PRISM benchmark suite. Out of five
runs of BRTDP-UCB only one finished (in 35 seconds) in the time limit
(240 seconds). TODO: comment on the rest. BMCTS did not perform well on
any of the benchmarks.

\autoref{table:branch_zconf} shows how the methods perform on
branch-zeroconf. TODO: there should be one where mcts is better, one
when vi is better and one where brtdp is better, let's wait for the
results.
Note that overall the benchmarks MCTS-BRTDP does
not perform the worst or the best anywhere, making it a universal choice.

Measurements show that the only method which works on at least some of the
composition-zeroconf models is value iteration (specifically with $K = 40$ the
model has 141 thousand states and VI solves it, regardless of choice of $N$).
Other methods always run out of time.
But on composition-wlan6 we observed that MCTS-BRTDP is the only method which
can actually solve the problem. Moreover, it has to explore only few
thousands of states and a few seconds of running time.

The impact of tree heuristic constant is usually trivial and we did not
observe any cases where it would be beneficial to fine-tune the
constant. As it forces exploration, significant increases beyond a small
useful value just increased the number of explored states. This can be
seen in table TODO.

\begin{landscape}

\begin{table}
\begin{tabularx}{\textwidth}{ l  | c | c | c | c | c }
                   & VI           &  BRTDP, MD       & MCTS-BRTDP-MD (0.5) & BRTDP-UCB (0.5)  \\
z-conf $N=15,  K=10$ & 205, 184k    &  0.93, 762       & 0.114, 819          &  12, 415         \\
z-conf $N=20,  K=10$ & 213, 184k    &  0.104, 769      & 0.110, 818          & 12.4, 436        \\
z-conf $N=20, K=20$  & -            &  0.129, 1172     & 0.172, 1208         & 12.7, 605        \\
z-conf $N=100, K=40$ & 16.2, 354k &  0.27, 2191        & 0.268, 2375         & -                \\
wlan6                &        204, &   \\
coin                 &        204, &   \\
leader               &        204, &   \\
\end{tabularx}
\caption{Comparison on standard PRISM models.}
\label{table:general_comparison}
\end{table}

\begin{table}
\begin{tabularx}{\textwidth}{ l  | c | c | c }
branch\_zeroconf &  BRTDP, MD  & MCTS-BRTDP-MD (0.5) &   VI      \\
$N=10,  K=10$      &   \\
$N=30,  K=10$      & \\
$N=140, K=40$      &
\end{tabularx}
\caption{Comparison on a ``branch'' model.}
\label{table:branch_zconf}
\end{table}

%\begin{table}
%\begin{tabularx}{\textwidth}{ l  | c | c
       %&  BRTDP, MAX-DIFF   &       VI     &  MCTS-BRTDP   & BRTDP-UCB \\
%wlan6  &             46,  &       204, &               & \\
%\end{tabularx}
%\caption{Composition models}
%\label{tab:composition}
%\end{table}
\end{landscape}

%\section{Model Structure and MCTS Performance}

%TODO: Explain why MCTS-BRTDP performs almost as well as BRTDP on some
%models

%We discuss what are the possible reasons MCTS performs well on some
%models and worse on others. (See e.g. On Adversarial Search Spaces and
%Sampling-Based Planning for a similar comparison between MCTS on Chess
%and Go).

%From the Chess and Go comparison it would seem that MCTS based methods would
%not perform well in verification as there are usually many dead ends.
%However by tuning the parameters and trying variants of the algorithms
%we have achieved better results on several models than other known methods.

%TODO :-/

%can we extrapolate where will each method do well from small instances of the models?

\chapter{Conclusion}

We introduced Markov decision processes,
the problem of their verification and known approaches to its solution.
Monte Carlo tree search was described in its most common variant UCT
(Upper Confidence bound applied to Trees),
together with its applications to maximizing rewards in MDPs and solving
games.
We suggested various new algorithms by combining the known
approaches to MDP verification with the techniques of MCTS. These
algorithms were implemented and evaluated on standard and new models.

We have observed the MAX-DIFF variant of BRTDP is a very strong
heuristic which itself often balances well between exploration and
exploitation in common MDP models which was our goal when designing the
MCTS based algorithm. Our MCTS-BRTDP algorithm
performs only slightly worse on the PRISM benchmark suite while it
performs significantly better on models hard for BRTDP. Even though
value iteration does well on such hard-for-BRTDP models it loses on the
PRISM suite, making MCTS-BRTDP a good universal choice for any MDP.

There is still a lot of work to be done in order to properly understand
how MCTS based methods may be applied in MDP verification. A
quantitative study of the algorithms' executions would help understand
which parts
of an MDP are explored even though they are not important and which
important parts are explored too late. This could be used to suggest new
formulas for tree node selection or other variations, however due to
complexity of the models this might be a very hard task.

Another interesting area of research might be into
new algorithms where the MCTS approach has better chances to
improve the search, for example one might try running MCTS and BRTDP in
stages, each time for a limited number of iterations, until the bounds
are sufficiently close.

There
are also rather easy practical tasks like adding support for time
bounded properties or extracting the strategy from the
solution\footnote{Once
the algorithm converges choose the action with the best upper bound at
each step.}.

%\printbibliography[heading=bibintoc] %% Print the bibliography.


\end{document}
