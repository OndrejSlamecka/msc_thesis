\chapter{Introduction}

A rover called Curiosity landed on Mars in August 2012 and is performing
research in the Gale crater since then. It has to make decisions with
uncertain consequences to maximize its scientific output but also to maintain
operational state.

Zero-configuration networking is a protocol for automatically setting up
computer networks without need for external help. The computers
select their IP addresses in the created network randomly with the goal
to achieve a working configuration.

Science models these and many similar real-world problems with {\em
Markov Decision Processes} (MDPs). The participants in these processes take
actions of their choice but the results of these actions are not
certain -- the rover may finish an experiment with high probability but
also break with small, the network may be configured with high
probability or not with small.

Various properties of Markov Decision Processes have been well studied
in the last seventy years. The usual target is to attain the highest
reward from the process, e.g. research as much as possible on Mars
\parencite{MarsRover}.
However in many situations it is desirable to learn what's the
probability the process fails (rover breaks) or succeeds (the network
configures) -- to avoid taking such actions, prevent
adversaries from forcing them or, for the positive case, to encourage them.

Classical algorithms for MDP verification (the act of proving
properties) based on dynamic programming can be used for finding the
maximum probability of failure (maximum over all possible decisions in
each step -- strategies). Under some circumstances these algorithms are
very good but recently it has been shown that learning based methods
like BRTDP can outperform them on many MDPs \parencite{atva14}.

Monte Carlo Tree Search is a heuristic search algorithm which has been
successfully used to find high reward strategies in Markov decision
processes. Recently the algorithm had a big success in the field of
computer Go.

In this thesis we explore how can Monte Carlo Tree Search be used to
find strategies maximizing given properties. The result are three
algortihms: one a variation of Upper Confidence Tree algorithm (an MCTS
variant), one a fusion of UCT and BRTDP and the last one a BRTDP variant
using a part of the UCT algorithm idea.  Measurements show that these
algorithms have advantage on several models.

The thesis is structured as follows. After this introduction, the second
chapter is devoted to Markov Decision Processes and prior work on their
verification.
The third chapter describes Monte Carlo Tree Search and its application
to maximizing rewards in MDPs and games. In the fourth chapter our new
algorithms are described and (TODO: PROVEN PAC). In the fifth chapter
the algorithms are evaluated on models available with the PRISM project,
moreover we try to isolate the properties of the models
which make the algorithms fast or slow. The sixth chapter concludes the
results with suggestions for future work.

The results in the fourth and fifth chapters are original results of
collaboration of Pranav Ashok, Tomáš Brázdil, Jan Křetínský and the
author of this thesis. The MCTS-BRTDP algorithm was suggested by the
author of this thesis and the BMCTS and BRTDP-UCB algorithms by Pranav
Ashok. The algorithms were implemented as part of the PRISM model
checker mostly when pair programming with Pranav Ashok.
